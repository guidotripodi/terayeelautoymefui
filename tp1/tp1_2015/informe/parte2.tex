
\subsection{Ejercicios}
\begin{itemize}
 \item 
\textbf{Ejercicio 3}  Completar la implementacion del scheduler Round-Robin implementando los
metodos de la clase SchedRR en los archivos sched rr.cpp y sched rr.h. La implementacion
recibe como primer parametro la cantidad de nucleos y a continuacion los valores de sus
respectivos quantums. Debe utilizar una unica cola global, permitiendo ası la migracion de
procesos entre nucleos.
\item \textbf{Ejercicio 4} Diseñar uno o mas lotes de tareas para ejecutar con el algoritmo del ejercicio
anterior. Graficar las simulaciones y comentarlas, justificando brevemente por que el comportamiento 
observado es efectivamente el esperable de un algoritmo Round-Robin.
\item \textbf{Ejercicio 5} A partir del articulo:\\
\begin{itemize}
 \item Liu, Chung Laung, and James W. Layland. Scheduling algorithms for multiprogramming
in a hard-real-time environment. Journal of the ACM (JACM) 20.1 (1973): 46-61.
\end{itemize}
1. Responda:
\begin{enumerate}
 \item ¿Que problema estan intentando resolver los autores?
 \item ¿Por que introducen el algoritmo de la seccion 7? ¿Que problema buscan resolver
con esto?
\item Explicar coloquialmente el significado del teorema 7.
\end{enumerate}
2. Disenar e implementar un scheduler basado en prioridades fijas y otro en prioridades
dinamicas. Para eso complete las clases $SchedFixed$ y $SchedDynamic$ que se encuentran
en los archivos $sched$\_$fixed.[h|cpp]$ y $sched$\_$dynamic.[h|cpp]$ respectivamente.
\end{itemize}


\subsection{Resultados y Conclusiones}

\subsubsection[Resolución Ejercicio 3]{Ejercicio 3}
Para desarrollar la implementacion del scheduler $Round-Robin$ y que este funcione de una forma correcta
utilizamos una serie de estructuras puntuales. \\
Las mismas son las siguientes:\\
\begin{enumerate}
 \item Una cola global, la cual nombramos $q$, esta contiene los $PID$ de los procesos activos que no estan
 bloqueados y en el tope de la misma se encuentra el proximo proceso a correr. Esta cola,
 fue desarrollada para que cuando se desaloje un proceso por finalizar su $quantum$ la misma pase al final de
 la cola y generando el ciclo acorde al comportamiento de este scheduler.
 \item Un vector denominado $cores$, este tiene en su elemento $i$ el pid correspondiente a
al proceso que está corriendo en el core $i+1$. Inicializamos todos los elementos en -1, esto
corresponde a la Idle Task, de esta forma reconocemos que no se cargaron procesos en los núcleos.
\item Un vector $quantum$ guarda en la posicion $i$ el quantum que se dispuso a cada núcleo.
\item Un vector $quantumActual$ aqui guardaremos la cantidad de ticks que le quedan al proceso
desde que fue cargado en el core.
\item Una lista de $bloqueados$ esta tendra procesos que se bloquearon cuando estaban corriendo.
\end{enumerate}

De esta manera, con estas estructuras nos permiten determinar para cada tarea, cuando, y cuanto 
de su quantum consumio de forma que podamos desalojarla correctamente.\\

A su vez, tomamos ciertas decisiones en esta implementación:
\begin{itemize}
 \item Si una tarea se encuentra bloqueada cuando se produce el tick del reloj, esta misma es desalojada
de la cola global, y agregada en un lista de bloqueados. A su vez, sera reseteado el quantum, se le
dara inicio a la proxima tarea que se encuentre ready y cuando el sistema operativo, nos envie una
señal de unblock, la tarea desalojada regresara al final de la cola global.

\end{itemize}


\subsubsection[Resolución Ejercicio 4]{Ejercicio 4}

\subsubsection[Resolución Ejercicio 5]{Ejercicio 5}

\textbf{PARTE 2}\\

\textbf{Algoritmo Scheduler Fixed - Explicacion de Implementacion}\\

Luego del estudio del Paper solicitado se realizo la implementacion del dicho algoritmo,
utilizando una serie de estructuras para que este funcione acorde a lo pedido:

\begin{itemize}
 \item Un struct denominado $tarea$ el cual contiene el pid, el run\_time\_actual, periodo dandonos
 informacion de la tarea cargada.
 \item Una lista de $tarea$ nombrada $tareas$, esta es ordenada segun el periodo de las tareas
 de menor a mayor, obteniendo la prioridad del algoritmo.
 \item Una variable global llamada $mayor$ en la cual guardamos el pid de la tarea con mayor
 periodo.
 \item Una variable global $primera\_pasada$ con la cual podemos realizar nuestra primer 
 comparacion de periodos entre tareas.
\end{itemize}

Ademas de estas estructuras utilizamos una funcion privada $insertarOrdenado$ la cual como el
nombre lo dice va guardando en nuestra lista de tareas, las tareas a corde se van cargando en su
respectivo lugar comparando los periodos de las mismas.\\
De esta forma, la secuencia del algoritmo fue la siguiente:

\begin{enumerate}
 \item Llega una nueva tarea se la carga e inserta ordenadamente en la lista.
 \item La tarea corre hasta finalizar.
 \item Una vez que finaliza se chequea en la lista cual es la primera en orden de prioridad queda
 esta ready para correr.
 \end{enumerate}

De esta forma, con las estructuras y funciones logramos que nuestro scheduler trabaje de la forma
deseada.
